\documentclass[twocolumn]{article}
\usepackage{amsmath}
\usepackage[pdfusetitle]{hyperref}
\usepackage[a4paper, total={6in, 8in}]{geometry}
\usepackage{microtype}

\title{STAT PHYS SCIENCE Midterm Cheat Sheet}
\author{Noppakorn Jiravaranun}
\date{\today}

\begin{document}
    \maketitle
    \section{Measure of Central Tendency}
    \subsection{Mean}
    \begin{equation}
        \mu = \frac{\sum_{i=1}^{N}x_{i}}{N}
    \end{equation}
    \begin{equation}
        \bar{x} = \frac{\sum_{i=1}^{n}x_{i}}{n}
    \end{equation}
    \subsection{Median}
    \begin{equation}
        M = x_{(\frac{N+1}{2})}
    \end{equation}
    \section{Quantiles}
    \subsection{Quartiles}
    \begin{equation}
        Q_{i} = x_{(i(\frac{N-1}{4}))}
    \end{equation}
    \subsection{Deciles}
    \begin{equation}
        D_{i} = x_{(i(\frac{N-1}{10}))}
    \end{equation}
    \subsection{Percentiles}
    \begin{equation}
        P_{i} = x_{(i(\frac{N-1}{100}))}
    \end{equation}

    \newpage

    \section{Measurement of Variation or Dispersion}
    \subsection{Range}
    \begin{equation}
        R = x_{max} - x_{min}
    \end{equation}
    \subsection{Average Deviation (A.D.)}
    \begin{equation}
        A.D. = \frac{\sum_{i=1}^{n} |x_{i}-\mu|}{n}
    \end{equation}
    \subsection{Standard Deviation (S.D.)}
    \subsubsection{Standard Deviation (Population)}
    \begin{equation}
        \sigma = \sqrt{\frac{\sum_{i=1}^{N} (x_{i}-\mu)^2}{N}} = \sqrt{\frac{\sum_{i=1}^{N} x_{i} -N\mu^2}{N}}
    \end{equation}
    \subsubsection{Standard Deviation (Sample)}
    \begin{equation}
        s = \sqrt{\frac{\sum_{i=1}^{n} (x_{i}-\mu)^2}{n-1}} = \sqrt{\frac{\sum_{i=1}^{n} x_{i} -N\mu^2}{n-1}}
    \end{equation}
    \subsection{Quatile Deviation}
    \begin{equation}
        Q.D. = \frac{Q_{3} - Q_{1}}{2}
    \end{equation}

    \newpage

    \subsection{Skewness}
    \subsubsection{Skewness (Population)}
    \begin{equation}
        S_{k} = \sum_{i=1}^{N}\frac{[x_{i} - \mu]^{3}}{\sigma^{3}N}
    \end{equation}
    if $S_{k} = 0$ the data is normal\\
    else if $S_{k} > 0$ the data is skwed right\\
    else if $S_{k} < 0$ the data is skwed left
    \subsubsection{Skewness (Sample)}
    \begin{equation}
        s_{k} = \sum_{i=1}^{n}\frac{[x_{i} - \bar{x}]^{3}}{s^{3}n}
    \end{equation}
    if $-1 \leq s_{k} \leq 1$ the data is normal\\
    else if $s_{k} > 1$ the data is skwed right\\
    else if $s_{k} < -1$ the data is skwed left
    \subsection{Kurtosis}
    A measure of the peakedness of a distribution
    \subsubsection{Kurtosis (Population)}
    \begin{equation}
        K = \sum_{i=1}^{N}\frac{[x_{i} - \mu]^{4}}{\sigma^{4}N}
    \end{equation}
    if $K = 0$ the data is normal\\
    else if $K > 0$ the data is higher than normal\\
    else if $K < 0$ the data is lower than normal
    \subsubsection{Kurtosis (Sample)}
    \begin{equation}
        k = ...
    \end{equation}
    if $-1 \leq k \leq 1$ the data is normal\\
    else if $k > 1$ the data is higher than normal\\
    else if $k < -1$ the data is lower than normal

\end{document}

